Vier Gewinnt und verschiedene Lösungsverfahren sind bereits ausgiebig untersucht. Das Spiel wurde 1988 von James Dow Allen und Victor Allis unabhängig voneinander mit wissensbasierten Methoden schwach gelöst. Dabei wurde gezeigt, dass der Spieler, der den ersten Zug macht, immer gewinnen kann\cite{Allen.2010}\cite{Allis.1988}.

Inzwischen wurde das Spiel auch durch einen Brute-Force Ansatz stark gelöst. Bei der Lösung durch John Trump kam Alpha-Beta-Pruning zum Einsatz, wobei alle 4.531.985.219.092 legalen Zustände des Spiels untersucht und auf ihre Gewinnchancen bewertet wurden\cite{Tromp}.

Lösungen, die alle Möglichkeiten durchrechnen, sind für den Einsatz in der Praxis aufgrund des hohen Rechenaufwands bei komplexeren Anwendungen selten praktikabel. Aus diesem Grund wird bevorzugt auf gute Heuristiken zurückgegriffen, die den Rechenaufwand minimieren, aber dennoch gute Ergebnisse liefern\cite{Heineman.October2008}.

Verschiedene allgemeine algorithmische und RL-basierte Ansätze wurden am Beispiel von Vier Gewinnt auf ihre Leistung untersucht. Es wurde gezeigt, dass sich sowohl algorithmische als auch RL-Ansätze bei Vier Gewinnt eignen\cite{Alderton.2019}\cite{Thill.2012}\cite{Wäldchen.2022}\cite{Taylor.2024}\cite{Sheoran.2022}\cite{Qiu.2022}.
