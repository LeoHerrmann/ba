Fortschreitende Automatisierung durchdringt zahlreiche Bereiche der Gesellschaft, so zum Beispiel die Fertigungsindustrie, das Gesundheitswesen oder den Straßenverkehr. Zwei fundamentale Ansätze sind dabei regelbasierte Algorithmen und Machine Learning. Die Einsatzbedingungen von Automatisierungssystemen unterscheiden sich häufig von den Bedingungen, unter denen sie entwickelt und getestet werden. Häufig müssen Systeme mit fehlerhaften oder veralteten Informationen arbeiten oder es treten Situationen ein, die bei der Konzipierung der Systeme nicht berücksichtigt werden können. Dabei sinkt die Leistungsfähigkeit dieser Systeme.

Im Rahmen dieser Arbeit werden Robustheit und Generalisierbarkeit eines algorithmischen Ansatzes und eines Reinforcement Learning basierten Ansatzes zur Lösung des Brettspiels \glqq Vier Gewinnt\grqq{} untersucht. Bei Robustheit und Generalisierbarkeit handelt es sich um Eigenschaften, die beschreiben, wie gut ein Algorithmus oder RL-Modell in der Praxis funktioniert, in der andere Bedingungen herrschen können als während der Entwicklung und Qualitätssicherung. Diese Kriterien sind besonders relevant für den Erfolg von Algorithmen und Modellen in der Praxis.

Spiele eignen sich zur Untersuchung von Algorithmen und Modellen, weil sie reale Probleme auf kontrollierbare Umgebungen abstrahieren und gleichzeitig reproduzierbare und vergleichbare Messungen ermöglichen. Die Untersuchungen dieser Arbeit erfolgen am Beispiel des Brettspiels \glqq Vier Gewinnt\grqq{}, da aus früheren Untersuchungen ersichtlich wird, dass sich sowohl algorithmische als auch Reinforcement Learning basierte Lösungen eignen.

Es wird Grundlagenforschung zu verbreiteten algorithmischen Ansätzen und Reinforcement Learning basierten Ansätzen betrieben. Anschließend werden die Aspekte Robustheit und Generalisierbarkeit von zwei Ansätzen aus den jeweiligen Bereichen am Beispiel von Vier Gewinnt empirisch untersucht. Dabei werden neue Erkenntnisse über Lösungsansätze von Vier Gewinnt gewonnen, die sich auf vergleichbare Szenarien in der realen Welt übertragen lassen.

Die zentrale Fragestellung lautet: Inwiefern sind bei Vier Gewinnt algorithmische oder Reinforcement Learning basierte Ansätze robuster oder besser generalisierbar? Das Ziel dieser Arbeit besteht darin, ein detailliertes Verständnis über verschiedene Aspekte von Robustheit und Generalisierbarkeit der zu untersuchenden Ansätze zu bekommen.