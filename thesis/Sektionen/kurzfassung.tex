In dieser Arbeit wird die Robustheit von symbolischen Algorithmen und Reinforcement-Learning-Verfahren miteinander verglichen. Dies geschieht am Beispiel des Brettspiels Vier Gewinnt. Dazu wurden zwei Agenten implementiert, die das Spiel selbstständig spielen. Beim ersten Agenten kommt der symbolische Algorithmus Monte Carlo Tree Search (MCTS) zum Einsatz. Dem zweiten Agenten liegt das RL-Verfahren Proximal Policy Optimization (PPO) zugrunde. Die Robustheit der Agenten wird quantifiziert, indem der Verlust der Gewinnrate gegen einen zufällig spielenden Agenten gemessen wird, während die zu untersuchenden Agenten zwei verschiedenen Szenarien mit ungünstigen Bedingungen ausgesetzt sind, die auf verschiedene Aspekte von Robustheit abzielen. Im ersten Szenario liegt Unsicherheit bezüglich Aktionen vor, was bedeutet, dass die Agenten keine vollständige Kontrolle darüber haben, in welche Spalte des Spielfelds sie ihre Spielsteine platzieren. Im zweiten Szenario erhalten die Agenten fehlerhafte Informationen über das Spielfeld, somit liegt Unsicherheit bezüglich Beobachtungen vor.

Es wird gezeigt, dass unter den beiden implementierten Agenten der MCTS-Agent im Szenario Unsicherheiten bezüglich Aktionen robuster ist als der PPO-Agent. Für das Szenario Unsicherheiten bezüglich Beobachtungen wurde dies bis zu einem gewissen Ausmaß an Unsicherheiten ebenfalls beobachtet. Die Ergebnisse dieser Arbeit lassen sich jedoch nicht auf den Vergleich von Robustheit zwischen MCTS und PPO unabhängig von der konkreten Implementierung oder dem Anwendungsfall übertragen, da für einen fairen Vergleich ein wesentlich aufwändigeres Training des PPO-Agenten notwendig gewesen wäre. Dafür und um allgemeine Aussagen über den Vergleich der Robustheit zwischen symbolischen Algorithmen und Reinforcement Learning zu übertragen, sind weitere Untersuchungen erforderlich.

\newpage