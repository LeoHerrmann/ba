Fortschreitende Automatisierung durchdringt zahlreiche Bereiche der Gesellschaft, so zum Beispiel die Fertigungsindustrie, das Gesundheitswesen oder den Straßenverkehr. Zwei fundamentale Ansätze sind dabei symbolische Algorithmen und Machine Learning (ML). Die Einsatzbedingungen von Automatisierungssystemen unterscheiden sich häufig von den Bedingungen, unter denen sie entwickelt und getestet werden. Häufig müssen Systeme mit fehlerhaften oder veralteten Informationen arbeiten oder es treten Situationen ein, die bei der Konzipierung der Systeme nicht berücksichtigt werden können. Dabei sinkt die Leistungsfähigkeit dieser Systeme.

Bei Robustheit handelt es sich um eine Eigenschaft von Software, die beschreibt, wie gut sie unter genau solchen veränderten Bedingungen funktioniert. Robustheit ist daher besonders relevant für den Erfolg von Algorithmen und Modellen in der Praxis und wird in der Literatur ausgiebig untersucht \cite{Micskei.2012}\cite{Moos.2022}\cite{Ni.2021}.

Bei der Entscheidung zwischen symbolischen Algorithmen und ML-Verfahren sind entscheidende Faktoren in der Regel die Leistungsfähigkeit, Zuverlässigkeit und Komplexität der Lösung und die Nachvollziehbarkeit der Lösungsfindung (\cite{Dabas.2022}; \cite{Ferguson.January2019}, Kapitel 1.1.2; \cite{Humm.2020}, S. 12 f.; \cite{Früh.2022}, S. 5f.). Hierbei stellt sich die Frage, inwiefern bei der Entscheidung zwischen symbolischen und ML-basierten Lösungsansätzen auch Robustheit ein relevantes Kriterium ist, und falls ja, welche Art von Verfahren robuster ist.

Spiele eignen sich zur Untersuchung von Algorithmen und Modellen, weil sie reale Probleme auf kontrollierbare Umgebungen abstrahieren und gleichzeitig reproduzierbare und vergleichbare Messungen ermöglichen. Die Untersuchungen dieser Arbeit erfolgen am Beispiel des Brettspiels Vier Gewinnt, da aus früheren Untersuchungen ersichtlich wird, dass sich dafür sowohl symbolische Lösungsansätze als auch Verfahren aus dem ML-Teilbereich Reinforcement Learning eignen \cite{Qiu.2022}\cite{Sheoran.2022}\cite{Alderton.2019}\cite{Taylor.2024}\cite{Dabas.2022}\cite{Wäldchen.2022}.

Im Rahmen dieser Arbeit wird zunächst Grundlagenforschung zu verbreiteten symbolischen und Reinforcement Learning basierten Ansätzen betrieben. Dabei wird aus beiden Bereichen jeweils ein Verfahren ausgewählt, das sich zur Lösung von Vier Gewinnt und zur Untersuchung der Fragestellung eignet. Diese beiden Verfahren werden anschließend am Beispiel von Vier Gewinnt empirisch auf deren Robustheit untersucht. Dadurch sollen neue Erkenntnisse darüber gewonnen werden, wie Robustheit als Kriterium für Entscheidungen zwischen symbolischen Algorithmen und ML eingesetzt werden kann.
