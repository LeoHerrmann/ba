Technikethik ist das Teilgebiet der Ethik, in dem Entscheidungen in den Bereichen Forschungsförderung, Regulierung und Anwendung von Technologien unter moralischen und gesellschaftlichen Gesichtspunkten beurteilt werden\footnote{\cite{Grunwald.2021}, S. 5}. Zu diesen Entscheidungen gehören beispielsweise, welche Arten von Technologien erlaubt sein dürfen oder welche Pflichten mit der Nutzung bestimmter Technologien einhergehen sollten\footnote{\cite{Nyholm.2023}, Kapitel 2.6}.

In der Technikethik kommen bewährte ethische Methoden zum Einsatz. Beispielsweise können durch einen Ausschuss aus Experten Richtlinien zu einem bestimmten Thema festgelegt werden. So sind unter anderem die \glqq Ethik-Leitlinien für eine vertrauenswürdige KI\grqq\footnote{\cite{EuropeanCommission:EthicsguidelinesfortrustworthyAI.2019}} von der Europäischen Kommission entstanden\footnote{\cite{Nyholm.2023}, Kapitel 3.3}.

Empirische Methoden können ebenfalls angewandt werden. Sie eigenen sich, um ein Verständnis für Muster in Sichtweisen von gewöhnlichen Menschen auf ein Thema zu bekommen\footnote{\cite{Nyholm.2023}m Kapitel 3.5}. Diese Muster wiederum können in ethische Diskussionen mit aufgenommen werden.

Auch Methoden der normativen Ethik werden kommen in der Technikethik zum Einsatz. Normative Ethik wird verwendet, um über bestimmte Handlungen Urteile zu fällen und zu begründen\footnote{\cite{Duwell.2011}, S. 25}. Daher eignen sich diese besonders im Rahmen dieser Hausarbeit zur ethischen Umgangs mit Automatisierung. Dazu gehören unter anderem konsequentialistische Theorien, bei denen Handlungen auf Grundlage ihres Nutzens und ihrer Folgen bewertet werden. Eine Handlung gilt dann als gut, wenn ihre positive Folgen größer sind als deren negativen Auswirkungen. Utilitarismus ist eine spezielle Ausprägung des Konsequentialismus, der Handlungen genau dann als gut bewertet, wenn das daraus resultierende Glück im Verhältnis zum daraus entstehenden Leid maximiert wird\footnote{\cite{Nyholm.2023}, Kapitel 2.5}.

Im Gegensatz zu den konsequentialistischen Theorien stehen die deontologischen Theorien\footnote{\cite{Grunwald.2021}, S. 171}, die Handlungen nicht in Abhängigkeit von ihren Konsequenzen bewerten, sondern anhand bestimmter moralischer Pflichten und Prinzipien\footnote{\cite{Neuhauser.2023}, S. 67}. Ein bekanntes Prinzip ist die goldene Regel, nach der Menschen immer so behandelt werden sollten, wie man auch selbst behandelt werden möchte\footnote{\cite{Baggini.2014}}. Der kategorische Imperativ von Immanuel Kant geht dabei noch einen Schritt weiter. Nach dem kategorischem Imperativ, soll immer so gehandelt werden, dass aus der Handlung ein Gesetz hervorgeht, bei dem man sich wünscht, dass sich alle Menschen daran halten würden\footnote{\cite{Neuhauser.2023}, S. 67}.

Durch den Einsatz von verschiedenen Methoden kann ein differenziertes Bild zu einem Thema geschaffen werden. 