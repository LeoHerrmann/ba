\label{reinforcement-learning}

RL ist ein Teilgebiet von Machine Learning. Beim Machine Learning geht es darum, Vorhersagen oder Entscheidungen zu treffen, indem ein Lösungsmodell eingesetzt wird, das automatisiert durch Beispieldaten generiert (trainiert) wurde. Im Gegensatz zu symbolischen Algorithmen muss das Verhalten des Lösungsmodells nicht explizit durch Menschen definiert werden. Machine Learning eignet sich daher für Probleme, für die es besonders schwierig ist, explizite Lösungsstrategien zu definieren (\cite{Humm.2020}, S. 12). Das Ziel beim RL besteht darin, für eine Umgebung, in der sich aufeinanderfolgende Entscheidungen gegenseitig beeinflussen, ein Regelwerk zu generieren, das den möglichen Zuständen der Umgebung die erfolgsversprechendsten Entscheidungen zuordnet. Beim RL wird das Lösungsmodell trainiert, indem es mit der Umgebung interagiert, und die Rückmeldung der Umgebung verarbeitet, um sein Regelwerk zu verbessern. Durch RL zu lösende Probleme werden häufig durch MDPs modelliert (\cite{Russell.2020}, S. 789 f.; \cite{Sutton.2018}, S. 1 f.). Reinforcement Learning ist nicht nur zur Lösung von Spielen verbreitet, sondern findet auch in Bereichen der Robotik Anwendung bis hin zur Personalisierung von Inhalten auf Webseiten (\cite{Russell.2020}, S. 850; \cite{Sutton.2018}, S. 450).

\subsubsection{Taxonomie}

Es existieren viele verschiedene Arten von RL-Verfahren. Dieses Kapitel beleuchtet weit verbreitete Kategorien, deren Eigenschaften und wie gut sich Verfahren aus diesen Kategorien zur Lösung von Vier Gewinnt und zur Beantwortung der Fragestellung eignen.

\paragraph{Tabellenbasierte vs. approximierende Verfahren}

\label{tabellenbasiert-vs-approximierende-verfahren}

Manche RL-Verfahren verwenden Tabellen, um die Grundlage für das Regelwerk abzubilden, andere Verfahren approximieren diese Tabellen. Bei tabellenbasierten Verfahren wie Q-Learning oder SARSA wird jedem Paar aus Zuständen und Aktionen ein Wert zugeordnet, der beschreibt, wie gut es ist, im jeweiligen Zustand die jeweilige Aktion zu wählen. Diese Verfahren eigenen sich für relativ kleine Zustandsräume mit einer Größe von bis zu $10^{6}$ Zuständen (\cite{Russell.2020}, S. 803 ff.). Es wurde sogar gezeigt, dass bei genügend Training die Leistung von Q-Learning-Agenten zu perfekten Verhalten konvergiert (\cite{Sutton.2018}, S. 140). Vier Gewinnt hat allerdings eine wesentlich höhere Zustandskomplexität von $10^{14}$ \cite{Allis.1994}. Um für jedes Paar aus Zuständen und Aktionen auch nur einen Bit zu speichern, wären $\frac{7}{8} \, \text{Byte} \cdot 10^{14} = 87.5 \, \text{Terabyte}$ Speicher erforderlich, und ein akkurates Modell zu trainieren würde zu viel Zeit in Anspruch nehmen (\cite{Russell.2020}, S. 803, \cite{Sutton.2018}, S. 195). In solchen Fällen muss die Tabelle approximiert werden. Dazu haben sich tiefe neuronale Netzwerke (DNNs) als etablierte Lösung herausgestellt. Wenn bei RL DNNs zum Einsatz kommen, spricht man von Deep RL (\cite{Russell.2020}, S. 809; \cite{Sutton.2018}, S. 236).

\paragraph{Modellbasierte vs. modellfreie Verfahren}

Bei RL wird zwischen modellbasierten und modellfreien Ansätzen unterschieden. Dabei bezieht sich der Begriff \glqq Modell\grqq{} nicht auf das Lösungsmodell, das bei beiden Ansätzen trainiert wird, sondern auf ein Modell der Umgebung, dass bei beim Training und der Nutzung von modellbasierten Methoden eingesetzt wird, um Vorhersagen über die Auswirkungen von Entscheidungen zu treffen. Modellfreie Methoden hingegen kommen ohne ein solches Modell aus. Der Agent lernt alleine durch die Interaktion mit der Umgebung und die dadurch erhaltene Rückmeldung (\cite{Russell.2020}, S. 790; \cite{Sutton.2018}, S. 7). Es ist anzumerken, dass alle in Kapitel \ref{symbolische-algorithmen} vorgestellten symbolischen Algorithmen ähnlich wie modellbasierte RL-Verfahren auf Modelle zurückgreifen, um Vorhersagen über das Verhalten der Umgebung zu treffen.

Daher sind modellfreie Methoden einfacher in der Implementierung und gut geeignet für Szenarien, die aufgrund ihrer Komplexität schwierig zu modellieren sind (\cite{Sutton.2018}, S. 12). Aufgrund der Fähigkeit, Vorhersagen über die Umgebung treffen zu können, weisen modellbasierte Methoden eine höhere Sample Complexity auf, was bedeutet, dass beim Training weniger Versuche benötigt werden, um ein effektives Regelwerk zu erlernen. Das ist besonders vorteilhaft, wenn Versuche teuer sind und nicht es eine Herausforderung darstellt, genügend Daten zu erheben, so zum Beispiel beim Training in der realen Welt (\cite{Russell.2020}, S. 687, S. 818, S. 959 f.).

Aufgrund des niedrigeren Implementierungsaufwands und des im Fall von Vier Gewinnt günstigen Trainings, richtet sich der Fokus der Arbeit auf modellfreie Methoden. Außerdem wurde in verschiedenen Untersuchungen modellfreie Methoden erfolgreich zur Implementierung von Agents für Vier Gewinnt eingesetzt \cite{Alderton.2019}, \cite{Taylor.2024}, \cite{Dabas.2022}, \cite{Wäldchen.2022}.

\paragraph{Wertbasierte vs. strategiebasierte Verfahren}

Modellfreie RL-Verfahren lassen sich in wertbasierte und strategiebasierte Varianten einteilen. Bei wertbasierten Verfahren wird eine Nutzenfunktion gelernt, die jedes Zustands-Aktionspaar bewertet. Bei der Anwendung eines trainierten wertbasierten Modells kommt eine Regelwerk zum Einsatz, das entsprechend der erlernten Nutzenfunktion für jeden Zustand immer die Aktion mit der besten Bewertung wählt (\cite{Russell.2020}, S. 790).

Bei strategiebasierten Methoden wird das Regelwerk nicht aus einer Nutzenfunktion abgeleitet, sondern das Regelwerk wird direkt erlernt (\cite{Russell.2020}, S. 790). Gegenüber wertbasierten Methoden hat das den Vorteil, dass dadurch auch Regelwerke modelliert erlernt werden können, die Entscheidungen basierend auf Wahrscheinlichkeiten treffen (\cite{Albrecht.2024}, S. 195). Ein klassischer Anwendungsfall ist das Spiel Schere-Stein-Papier, bei dem das optimale Regelwerk darin besteht, alle Aktionen (Schere, Stein, Papier) zufällig mit derselben Wahrscheinlichkeit zu wählen. Ein solches wahrscheinlichkeitsbasiertes Regelwerk kann durch wertbasierte Methoden nicht abgebildet werden, da sie stets den einen laut Werte-Funktion vermeintlich besten Zustand wählen. Ein weiterer Vorteil von strategiebasierten Methoden ist, dass sie kontinuierlichen Aktionsräume abbilden können und wertbasierte Methoden nicht (\cite{Albrecht.2024}, S. 196).

Im Fall von Vier Gewinnt sind beide Vorteile von strategiebasierten Verfahren nicht relevant, da Vier Gewinnt einen diskreten Aktionsraum besitzt, und zufälliges Handeln keinen strategischen Vorteil bringt.

\paragraph{Single-Agent vs. Multi-Agent Reinforcement Learning}

Vier Gewinnt kann als Problem des Gebiets Multi-Agent RL (MARL) betrachtet werden. MARL ist ein Teilgebiet des RL, in denen mehrere RL-Agenten in derselben Umgebung miteinander interagieren (\cite{Albrecht.2024}, S. 2). Die Agenten können in der Umgebung ein kompetitives oder kooperatives Verhältnis oder eine Mischung beider Verhältnisse zueinander haben (\cite{Albrecht.2024}, S. 9). In Zwei-Spieler Nullsummenspielen wie Vier Gewinnt arbeiten die Agenten rein kompetitiv. Ein entscheidender Unterschied von kompetitiven MARL-Problemen zu Single-Agent-RL-Problemen (SARL) besteht darin, dass in SARL-Problemen die Umgebung eines trainierenden Agents statisch ist, was bedeutet, dass sich die Übergangsfunktion des zugrundeliegenden MDPs nicht ändert. Beim Training in einer Multi-Agent-Umgebung lernen mehrere Agenten gleichzeitig, damit ändert sich die Übergangsfunktion und die Umgebung ist nicht statisch. Die Agents müssen sich im Trainingsprozess an die sich ändernde Umgebung anpassen können (\cite{Albrecht.2024}, S. 12).

Es gibt MARL-Methoden, die auf eine sich ändernde Umgebung optimiert sind. Dazu gehören Beispielsweise Methoden, die dem Konzept \glqq Centralized Training Decentralized Execution\grqq{} (CTDE) zuzuordnen sind. CTDE bedeutet, dass Agenten während des Trainings aus den Erfahrungen voneinander lernen, aber ihre Entscheidungen trotzdem selbstständig treffen können (\cite{Albrecht.2024}, S. 231). Da solche koordinierenden Ansätze zusätzliche Komplexität einführen, wird in dieser Arbeit der Fokus auf Independent Learning des Bereichs \glqq Decentralized Training Decentralized Execution\grqq{} gerichtet. Beim Independent Learning interagieren die Agenten zwar im Training miteinander, erlernen ihr Regelwerk jedoch unabhängig voneinander. Bei Nullsummenspielen geschieht Independent Learning üblicherweise im Zusammenhang mit Self-Play, was bedeutet, dass alle Agenten das selbe Lernverfahren einsetzen. Im Training lernen die Agenten, gegenseitig ihre Schwächen auszunutzen und diese Schwächen zu beheben. Es ist aber auch möglich, Mixed-Play anzuwenden, also im Training Agenten mit verschiedenen Lernverfahren gegeneinander antreten zu lassen. Über Independent Learning lassen sich auch SARL-Methoden auf MARL-Probleme anwenden. Dabei ist zu berücksichtigen, dass SARL-Modelle in nicht-stationären Umgebungen ein weniger stabiles Lernverhalten aufweisen als bei stationären Umgebungen, dennoch werden sie in der Praxis häufig erfolgreich für MARL-Probleme eingesetzt (\cite{Albrecht.2024}, S. 221 f.).

Außerdem ist anzumerken, dass sich Off-Policy-Verfahren weniger für MARL eigenen als On-Policy-Verfahren, weil Off-Policy-Verfahren Entscheidungen basierend auf Erfahrungen treffen, die mehrere Lernvorgänge in der Vergangenheit liegen, in der der Gegenspieler noch eine inzwischen veraltete Strategie hatte. Agents mit On-Policy Algorithmen hingegen lernen nur von anhand des letzten Lernvorgangs und damit der aktuellsten Strategie der anderen Agenten. Das kann zu stabilerem Lernverhalten führen (\cite{Albrecht.2024}, S. 224 f.).

\subsubsection{Künstliche neuronale Netzwerke}

\label{kuenstliche-neuronale-netzwerke}

Bei künstlichen neuronalen Netzwerken (KNN) handelt es sich um eine weit verbreitet Methode des Machine Learning zur Approximation von komplexen, nicht-linearen Funktionen (\cite{Albrecht.2024}, S. 164 f.). Wie in Kapitel \ref{tabellenbasiert-vs-approximierende-verfahren} erwähnt, werden neuronale Netzwerke im Zusammenhang mit Deep Reinforcement Learning eingesetzt, um eine Approximierung für die optimale Nutzenfunktion oder das optimale Regelwerk zu finden.

\paragraph{Aufbau}

Den Hauptbestandteil von künstlichen neuronalen Netzwerken bilden die Neuronen. Sie bestehen aus folgenden Komponenten:

\begin{itemize}
	\item Eingabewerte $x_1$ bis $x_n$
	\item Gewichte für jeden Eingabewert $w_1$ bis $w_n$
	\item Bias $b$
	\item Nicht-lineare Aktivierungsfunktion $g$
	\item Ausgabewert, der berechnet wird, indem die Gewichtete Summe aus den Eingabewerten mit dem Bias addiert wird, und dann in der Aktivierungsfunktion verrechnet wird (\cite{Albrecht.2024}, S. 166 f.).
\end{itemize}

In künstlichen neuronalen Netzwerken sind diese Neuronen schichtweise miteinander verbunden. In KNNs mit der Feedforward-Eigenschaft, auf die sich diese Arbeit beschränkt, nimmt jedes Neuron Ausgaben der Neuronen der vorangegangenen Schicht als Eingaben entgegen. Es gibt keine Rückkopplungen oder zyklischen Verbindungen.

Eine Ausnahme bilden die Neuronen der ersten Schicht, der sogenannten Eingabeschicht. Darin nimmt jedes Neuron als Eingabewert einen Parameter der zu approximierenden Funktion entgegen. Auf die Eingabeschicht folgen beliebig viele versteckte Schichten. Die Neuronen in den versteckten Schichten verarbeiten die Eingaben entsprechend nach Gewichten, Bias und Aktivierungsfunktion und leiten deren Ausgaben an die Neuronen in der nächsten Schicht weiter. Das passiert solange, bis die Ausgabeschicht erreicht wurde. Sie enthält so viele Neuronen, wie Ausgabewerte berechnet werden sollen (\cite{Albrecht.2024}, S. 165 f.; \cite{Russell.2020}, S. 751 f.).

Die Gewichte und Biase der Neuronen sind zunächst zufällig initialisiert und werden im Zuge des Trainings auf Grundlage von Beispieldaten optimiert, sodass das Netzwerk die Zielfunktion so gut wie möglich approximiert (\cite{Albrecht.2024}, S. 169).

% Variation des Aufbaus

Der Aufbau eines KNNs lässt sich unter anderem über die Anzahl der versteckten Schichten, der darin enthaltenen Neuronen, der Art, wie sie miteinander verbunden sind, und den eingesetzten Aktivierungsfunktionen variieren (\cite{Russell.2020}, S. 759).

Durch größere Netzwerke können komplexere Probleme gelöst werden, bei zu großen Netzwerken besteht jedoch die Gefahr des Overfitting, was bedeutet, dass das Netzwerk schlecht mit Eingaben umgehen kann, die es im Training nicht gesehen hat (\cite{Albrecht.2024}, S. 166; \cite{Sutton.2018}, S. 225). Es wurde außerdem gezeigt, dass KNNs bei gleicher Anzahl von Gewichten und Biases bessere Ergebnisse erzielen, wenn sie tiefer statt breiter sind, also mehr haben anstatt mehr Neuronen pro Schicht haben (\cite{Russell.2020}, S. 769).

Als Aktivierungsfunktion sind derzeit Rectified Linear Unit (ReLU) und Variationen davon verbreitet. ReLU gibt für Eingabewerte < 0 0 und ansonsten den Eingabewert zurück (\cite{Albrecht.2024}, S. 167 f.; \cite{Russell.2020}, S. 759).

Der optimale Aufbau eines KNNs hängt vom zu lösenden Problem ab. Es gibt Werkzeuge, die beim Finden eines guten Aufbaus unterstützen, dabei erfolgt dieser Prozess in der Praxis häufig auch durch Experimente und unter Zuhilfenahme von menschlicher Erfahrung und Intuition (\cite{Russell.2020}, S. 759).

\paragraph{Training}

Während des Trainings werden die zunächst zufällig initialisierten Parameter $\theta$ (Gewichte und Biase der Neuronen) so optimiert, dass das KNN die Zielfunktion möglichst gut approximiert (\cite{Albrecht.2024}, S. 169).

Dazu muss bestimmt werden können, wie gut das neuronale Netzwerke seine Aufgabe löst. Als Indikator dafür dient der Verlust. Sind die Ausgabewerte bekannt, die das KNN für bestimmte Eingaben liefern soll, so wie es im ML-Teilbereich Supervised Learning der Fall ist, kann der Verlust eines KNNs durch den Mean Squared Error, also die durchschnittliche quadrierte Differenz zwischen berechneten und tatsächlichen Werten angegeben werden (\cite{Albrecht.2024}, S. 170). Wie Verlust bei RL-Verfahren berechnet wird, hängt vom konkret eingesetzten Verfahren ab (\cite{Sutton.2018}, S. 225).

Der Verlust kann als Funktion $L(\theta)$ betrachtet werden, die von den Parametern des KNNs abhängt. Das KNN löst seine Aufgabe dann gut, wenn das Minimum dieser Verlustfunktion erreicht wurde. Um das Minimum zu finden, wird der Gradient (\glqq multidimensionale Ableitung \grqq{}) der Verlustfunktion $\Delta_\theta L(\theta)$ betrachtet, der die Steigung dieser Verlustfunktion beschreibt. Wird der Gradient für einen Satz von Parametern berechnet, kann daraus gefolgert werden, in welche Richtung und in welchem Verhältnis die Parameter zueinander verändert werden müssen, um den Verlust zu reduzieren (\cite{Albrecht.2024}, S. 171).

Auf dieser Tatsache beruht das Gradientenverfahren. Nach dem Gradientenverfahren, werden die Parameter $\theta$ wiederholt wie folgt angepasst, bis ein Minimum erreicht wurde:

\[ \theta \leftarrow \theta {-} \alpha * \Delta_\theta L(\theta) \]

$\alpha$ bezeichnet hierbei die Lernrate. Wird sie zu klein gewählt, erfolgt die Annäherung an das Minimum sehr langsam. Wenn sie zu groß gewählt wird, kann es passieren, dass erst gar kein Minimum gefunden wird (\cite{Ferguson.January2019}, Kapitel 10.3; \cite{Buduma.2022}, Kapitel 4). Daher existieren Verfahren, die die Lernrate dynamisch anpassen können, um den Optimierungsprozess zu beschleunigen (\cite{Albrecht.2024}, S. 174). Bei der Lernrate handelt es sich um einen Hyperparameter. Hyperparameter beeinflussen den Lernprozess und müssen vor dem Training gezielt initialisiert werden. Ähnlich wie die Gestaltung der Architektur erfolgt die Initialisierung der Hyperparameter häufig händisch, wobei automatisierte Werkzeuge dabei unterstützen können \cite{Feurer.2019} \cite{Shawki.2021}.

Um den Gradienten der Verlustfunktion für einen bestimmten Satz von Parametern zu berechnen, wird bei neuronalen Netzwerken das Verfahren Backpropagation angewendet. Auf Grundlage des Verlustes von jedem einzelnen im Training verfügbaren Datenpunkt wird rekursiv von der Output-Schicht bis hin zur Input-Schicht bestimmt, wie die Parameter im Verhältnis zueinander geändert werden müssen, um den Verlust zu reduzieren. Der Gradient entspricht dem Durchschnitt dieser Änderungen (\cite{Russell.2020}, S. 766 f., \cite{Albrecht.2024}, S. 174f.).

Das Standard-Gradientenverfahren berechnet den Verlust auf Grundlage aller beim Training verfügbaren Daten. Dies ist mit hohem Rechenaufwand verbunden. Aus diesem Grund existieren das stochastische Gradientenverfahren und das Mini-Batch-Gradientenverfahren, die den Verlust nicht basierend auf allen verfügbaren Daten, sondern nur auf Grundlage eines Datenpunktes bzw. einer Teilmenge der Daten berechnen (\cite{Albrecht.2024}, S. 172).

Es ist anzumerken, dass über das Gradientenverfahren in den meisten Fällen nicht das globale Minimum der Verlustfunktion gefunden werden kann, sondern nur ein lokales Minimum. Dies reicht in den meisten Fällen jedoch aus, da die lokalen Minima von Verlustfunktionen gerade bei größeren KNNs größtenteils ähnlich niedrige Werte aufweisen, und die Wahrscheinlichkeit, ein lokales Minimum mit einem wesentlichen höheren Wert zu finden, sehr niedrig ist (\cite{Sutton.2018}, S. 200; \cite{Ferguson.January2019}, Kapitel 5.4.4; \cite{Choromanska.2015}).

\subsubsection{Advantage Actor-Critic}

Advanced Actor Critic (A2C) ist ein weit verbreitetes Verfahren, das sich nach den vorangegangenen Kapiteln zur Lösung von Vier Gewinnt eignet. Es ist ein Single-Agent-Verfahren, das modellfrei und off-policy arbeitet, und parametisierte Funktionen (wie z.B. KNNs) einsetzt, um das optimale Verhalten zu approximieren. Darüber hinaus handelt es sich um ein Actor-Critic-Verfahren, welche strategiebasierte Ansätze in der Actor-Komponente mit wertebasierten Ansätzen in der Critic-Komponente kombinieren (\cite{Albrecht.2024}, S. 202 ff.).

\paragraph{Strategiebaierte Actor-Komponente}

Bei Actor-Critic-Verfahren kommen in der Actor-Komponente stets Policy-Gradient-Verfahren zum Einsatz. Policy Gradient Verfahren sind eine Unterart von strategiebasierten RL-Verfahren. Sie setzen voraus, dass das Regelwerk als parametrisierte Funktion abgebildet ist, so wie es beispielsweise bei KNNs der Fall ist. Denn dann gilt das Policy Gradient Theorem, das Aussagen über den Gradienten der Leistungsfähigkeit eines parametrisierten Regelwerks trifft (\cite{Sutton.2018}, S. 324; \cite{Albrecht.2024}, S. 195)

Das Policy-Gradient-Verfahren, das die Grundlage des Actors in A2C bildet, ist das REINFORCE-Verfahren (\cite{Albrecht.2024}, S. 203). Der Agent startet dabei an einem Startzustand der Trainingsumgebung und trifft dabei solange Entscheidungen anhand des Regelwerks $\pi$, bis ein Endzustand erreicht wurde. Damit ist eine Episode abgeschlossen. Dabei speichert er die Historie der in der Episode besuchten Zustände $S$, durchgeführten Aktionen $A$ und erhaltenen Belohnungen $R$. Für jeden Schritt $t$ in der Historie werden die Parameter $\phi$ des Regelwerks $pi$ im Rahmen des Gradientenverfahrens nach folgendem Ausdruck angepasst:

\[ \phi \leftarrow \phi {+} \alpha \gamma ^{t} * G_t * \frac{\Delta \pi (A_t|S_t, \phi)}{\pi (A_t|S_t, \phi)} \]

Anders als in Abschnitt \ref{kuenstliche-neuronale-netzwerke} wird hier allerdings nicht der Verlust reduziert, sondern es wird die Wahrscheinlichkeit erhöht, bei Zustand $S_t$ die Aktion $A_t$ auszuführen. Das geschieht proportional zur diskontierten Belohnung $G_t$, sodass die Wahrscheinlichkeit für Züge mit größerer Belohnung stärker erhöht wird. Der Gradient $\Delta \pi (A_t|S_t, \phi_t)$ zeigt dabei an, in welche Richtung und in welchem Verhältnis die Parameter zueinander verschoben werden müssen, um die Wahrscheinlichkeit, den Zug $A_t$ bei Zustand $S_t$ auszuführen, zu maximieren. Der Gradient wird durch die Wahrscheinlichkeit $\pi (A_t|S_t, \phi_t)$, den Zug auszuführen, dividiert, um dem Effekt entgegenzuwirken, dass Züge mit einer höheren Wahrscheinlichkeit häufiger gewählt werden, und damit die Gewichte in Richtung der Züge mit höherer Wahrscheinlichkeit überproportional verschoben werden würden.

$\gamma$ ist der Diskontierungsfaktor, für den gilt $0 <= \gamma <= 1$. Je kleiner, desto mehr beeinflussen frühere Züge die Parameter als spätere Züge.

Die diskontierten Belohnung $G_t$ errechnet sich dabei aus einer gewichteten Summe der in der Episode erhaltenen Belohnungen:

\[ G_t \leftarrow \sum_{k=t+1}^{T} \gamma ^{k-t-1} * R_k \]

Hier kommt erneut der Diskontierungsfaktor $\gamma$ zum Einsatz. Je größer der Diskontierungsfaktor, desto stärker werden Belohnungen gewichtet, die weiter in der Zukunft liegen. Je kleiner, desto \glqq kurzsichtiger\grqq{} ist der Agent (\cite{Sutton.2018}, S. 55).

Die Wahrscheinlichkeit, im Zustand $S_t$ den Zug $A_t$ auszuführen, ergibt sich über das Regelwerk selbst. Der Gradient davon wird über den Backpropagation-Algorithmus ausgerechnet (\cite{Sutton.2018}, S. 326 f.).

Die Idee, dass die Erhöhung der Wahrscheinlichkeit der Züge, auch die Leistungsfähigkeit des Regelwerks erhöhen wird, ist aus dem Policy Gradient Theorem hergeleitet (\cite{Sutton.2018}, S. 326 f.).

REINFORCE weist häufig langsames und instabiles Training auf. Das liegt daran, dass die Parameter auf Grundlage von den Entscheidungen einer gesamten Episode angepasst werden, die einer Wahrscheinlichkeitsverteilung unterliegen und damit eine hohe Varianz aufweisen (MARL, S. 200). Um dieser hohen Varianz entgegenzuwirken, wird bei A2C ein wertebasierter Critic eingesetzt.

\paragraph{Wertebasierte Critic-Komponente}

Bei der Critic-Komponente handelt es sich um eine parametrisierte Funktion $V(s, \theta)$, die unter Berücksichtigung der Parameter $\theta$ Schätzungen über den Wert eines gegebenen Zustands $s$ liefert (MARL, S. 202 ff.)

Bei A2C wird während des Trainings nach jeder durchgeführten Aktion ein Advantage-Wert berechnet. Dabei handelt es sich um die Differenz zwischen dem geschätzten Wert des Zustands, in dem sich der Agent befunden hat, bevor er die Aktion durchgeführt hat, und einem neuen Schätzwert, der sich aus der Summe der durch die Aktion erhaltene Belohnung und dem Schätzwert des über die Aktion erreichten neuen Zustands zusammensetzt:

\[ Adv(s_t, a_t) = r_t + \gamma V(s_{t+1}, \theta) - V(s_t, \theta) \]

Ein positiver Advantage-Wert bedeutet, dass der durch die Aktion erreichte neue Zustand höherwertiger ist als durch die Wertefunktion angenommen. Ein negativer Advantage-Wert bedeutet, dass er weniger wert ist.

Dementsprechend werden die Parameter der Wertefunktion unter Verwendung des Gradientenverfahrens aktualisiert. Der Verlust wird hierbei als Quadrat des Advantage-Wertes definiert (MARL, S. 205):

\[ L(\theta) = (r_t + \gamma V(s_{t+1}, \theta) - V(s_t, \theta))^2 \]

Es ist anzumerken, dass hierbei Endzustände einer gesonderten Betrachtung bedürfen (vgl. MARL, S. 205). Aus Gründen der Übersichtlichkeit wird in dieser Arbeit darauf verzichtet.

\paragraph{Actor- und Critic-Komponenten im Zusammenspiel}

Das Zusammenspiel zwischen Actor- und Critic-Komponente gestaltet sich bei A2C so, dass die Parameter der auf REINFORCE basierten Actor-Komponente nicht am Ende einer Episode proportional zur in der Episode erhaltenen Belohnung aktualisiert werden, sondern stattdessen werden die Parameter bei jedem Schritt im Training proportional zum über die Critic-Komponente berechneten Advantage-Wert aktualisiert (MARL, S. 205):

\[ \theta \leftarrow \theta + \alpha Adv(s_t, a_t) \frac{\Delta \pi (A_t|S_t, \theta_t)}{\pi (A_t|S_t, \theta_t)} \]

Dadurch, dass die Parameter der Funktionen in Abhängigkeit von den Erfahrungen aus einem Trainingsschritt und nicht einer gesamten Episode aktualisiert werden, weisen die Parameteraktualisierungen weniger Varianz auf. Kombiniert mit der dadurch häufigeren Anzahl an Parameteraktualisierungen führt dies in vielen Fällen zu effizienterem Training (MARL, S. 202).

Im Rahmen der Realisierung kommt eine optimierte Variante von A2C zum Einsatz, die sich Proximal Policy Optimization (PPO) nennt. Sie enthält unter anderem Mechanismen, die große Sprünge in den Veränderungen des Regelwerks verhindert, was zu einem effizienterem Trainingsverhalten und einer geringeren Sensibilität gegenüber Hyperparametervoreinstellungen führt (MARL, S. 206 ff.).
